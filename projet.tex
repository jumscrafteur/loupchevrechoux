\documentclass{article}
\usepackage{tikz}
\usepackage{fullpage}
\usepackage{amsmath,amsfonts,amssymb,stmaryrd}

\title{Projet de Logique: Loup Chèvre Chou}
\author{Waxin Alban, Le Meur Quentin, Lorin Guillaume, Sansané Hugo, Said Yacine}
\date{}

\setcounter{tocdepth}{5}
\setlength{\topmargin}{-1.5cm}
\setlength{\textheight}{25cm}
\setlength{\textwidth}{18cm}
\setlength{\oddsidemargin}{-1.5cm}
\setlength{\evensidemargin}{50cm}

\newcommand{\R}{\mathbb{R}}
\newcommand{\C}{\mathbb{C}}
\newcommand{\N}{\mathbb{N}}
\newcommand{\Q}{\mathbb{Q}}
\newcommand{\Z}{\mathbb{Z}}
\newcommand{\K}{\mathbb{K}}
\newcommand{\M}{\mathcal{M}}
\newcommand{\fct}[5]
	{
	  \begin{array}{ccccc}
		#1 & : & #2 & \to & #3 \\
	    && #4 & \mapsto & #5 \\
	  \end{array}
	}

\begin{document}
\maketitle
\tableofcontents \newpage
\section{Sujet}
\subsection{choix et règles}
L’énigme du loup, de la chèvre et du chou est une énigme de type “Passage de pont” dont les origines remontent au IXeme siècle.
Le but est de faire traverser 4 entités : un fermier , un loup, une chèvre et un chou d’une rive gauche à une rive droite à l’aide d’un bateau avec les contraintes suivantes :

\begin{itemize}
  \item  Il est impossible de transporter plus de 2 entités en même temps sur le bateau.
  \item  Le bateau doit être conduit par le fermier uniquement.
  \item  Il est impossible de laisser le loup et la chèvre seul sans le fermier.
  \item  Il est impossible de laisser la chèvre et le chou seul sans le fermier.
\end{itemize}

\subsection{Etudes préliminaires}
Il existe des variantes à l’énigme du loup, de la chèvre et du chou avec d’autres entités ou un nombre de place différents dans le bateau, nous nous concentrerons dans notre sujet sur la version classique de l’énigme puis nous émettrons des axes d’amelioration pour généraliser la solution.
\\Une étude préliminaire nous a permis de determiner les nécessités suivantes :
\begin{itemize}
  \item Notre logique doit permettre de modéliser l’aspect temporel du problème.
  \item Notre logique doit permettre de visualiser les états de chaque berges.
\end{itemize}

\section{Définition de la logique}
\subsection{Formalisation}
Nous utiliserons des fonctions constantes pour modéliser nos entités :
\begin{itemize}
  \item $f$ : pour l’entité Fermier
  \item $l$ : pour l’entité Loup
  \item $c$ : pour l’entité Chèvre
  \item $b$ : pour l’entité Chou
\end{itemize}
Nous utiliserons une infinité de prédicats représentant trois informations à un moment t donné:
\begin{itemize}
  \item $RG_t(x)$ : x est sur la rive gauche a l’itération t
  \item $RD_t(x)$ : x est sur la rive droite a l’itération t
  \item $S_t(x)$ : x est en sécurité a l’itération t
  \item $P_t$ : la partie est perdue à l'itération t
  \item $G_t$ : la partie est gagnée à l'itération t
\end{itemize}
Ainsi qu'un prédicat représentant la relation prédateur/proie :
$M(x,y)$ : x mange y.\\
Le bateau ne sera pas modélisé car il ne représente que la transition entre deux itérations.
\subsection{Langage}
Nous avons décidé de choisir une logique du Premier Ordre, car elle permet de décrire les personnages
à l’aide de termes d’arité 0 et les états des règles de sécurité par des prédicats \\
\\
Soit $\mathcal{L} = (\mathcal{F} ,\mathcal{P} ,\mathcal{C} ,\mathcal{Q} )$
\begin{itemize}
  \item $ \mathcal{F} := \{(f,0),(l,0),(c,0),(b,0)\}$
  \item $ \mathcal{P} := \{\forall i \in \N,(P_i,0),(G_i,0),(S_i,1),(RG_i,1),(RD_i,1),(M,2)\} $
  \item $ \mathcal{C} := \{(\bot , 0),(\top , 0),(\neg , 1), (\vee , 2), (\wedge , 2), (\rightarrow , 2)\}$
  \item $ \mathcal{Q} := \{\exists, \forall \}$
\end{itemize}
\subsection{Syntaxe}
Termes : \\
$t := x|f|l|c|b$ \\
Formules : \\
$A := \forall i \in\N, P_i|G_i|S_i(t)|RD_i(t)|RG_i(t)|M(t,t)|(t=t)| \bot|\top|\neg A|A\vee A|A\wedge A|A \rightarrow A$

\subsection{Notation}
$ a \leftrightarrow b = (a \rightarrow b) \wedge  (b \rightarrow a)$
\subsection{Interprétation}
Domaine d'Interprétation : $D = \{f,l,c,b\}$
\begin{itemize}

  \item $\fct{I(M)}{D^2}{\{\bot,\top\}}{(x,y)}{\neg(x=f) \wedge \neg(x=b) \wedge (x=l \leftrightarrow y=c) \wedge (x=c \leftrightarrow y=b)}$
  \item $\fct{I(S_t)}{D}{\{\bot,\top\}}{x}{(RG_t(x) \to (\forall y , RG_t(y) \to \neg M(y,x) )) \vee (RD_t(x) \to (\forall y , RD_t(y) \to \neg M(y,x) ))} $
  \item[$ \to $] État initial :
    \begin{itemize}
      \item[$ \to $] Rive Gauche :
      \item $ RG_0(f) = \top $
      \item $ RG_0(l) = \top $
      \item $ RG_0(c) = \top $
      \item $ RG_0(b) = \top $
      \item[$ \to $] Rive Droite
      \item $ RD_0(f) = \bot $
      \item $ RD_0(l) = \bot $
      \item $ RD_0(c) = \bot $
      \item $ RD_0(b) = \bot $
    \end{itemize}

\end{itemize}


\subsection{Axiomatisation}
Soit A, B et C des variables propositionnelles.

\begin{itemize}
  \item Tautologies. Les tautologies de la logique du premier ordre
        sont construites comme suit
  \item[] \begin{enumerate}
      \item
            Soit $A(p_1, \dots , p_n)$ une tautologie de la logique propositionnelle
      \item Prendre n formules de la logique du premier ordre $B_1, \dots , B_n$
      \item Substituer la variable propositionnelle $p_1$ par la formule $B_1$, \dots, $p_n$ par la formule $B_n$
      \item La formule $A(B_1, \dots , B_n)$ est une tautologie de la logique du premier ordre.
    \end{enumerate}
  \item Les axiomes des quantificateurs.
  \item[] \begin{itemize}
      \item $\exists x, A \leftrightarrow \neg \forall x, \neg A$
      \item $\forall x,(A \rightarrow B) \rightarrow (A \rightarrow \forall x, B)$ où x est une variable qui n’a pas d’occurrence libre dans A.
      \item $(\forall x, A) \rightarrow A_{t / x}$ où t est un terme tel que aucune occurrence libre de x ne se trouve dans le champ d'un quantificateur liant une variable de t
    \end{itemize}
\end{itemize}
\newpage
\subsection{Axiomes}
\begin{itemize}
  \item \small On définit une infinité d'axiomes représentants chacun l'état de sécurité d'une entité à un moment t donné:
        \begin{align*}
          a_1:= \forall x,S_t(x) & \leftrightarrow  (\forall y, \neg M(y,x)) \\ 
                                 &\vee [\exists y , M(y,x) \wedge ((RG_t(x)\wedge RD_t(y)) \vee (RD_t(x)\wedge RG_t(y)))]\\
                                 & \vee [RG_t(x) \wedge RG_t(f)]             \\
                                 & \vee [RD_t(x) \wedge RD_t(f)]
        \end{align*}
  \item On définit le fait que l'on ne puisse pas avoir un gain et une perte en même temps\\
        $a_2:= G_t \to \neg P_t$
  \item On définit ce qui rend un chemin perdant, c'est à dire avoir perdu à l'itération précédente ou avoir une entité qui n'est pas en sécurité à l'itération courante\\
        $a_3:= P_t \leftrightarrow (\exists x, \neg S_t(x) \vee P_{(t-1)})$
  \item On définit un chemin gagnant par le fait qu'à un moment on ait les quatres entités sur la rive droite\\
        $a_4:= G_t \leftrightarrow [RD_t(l) \wedge RD_t(f) \wedge RD_t(c) \wedge RD_t(b)] \leftrightarrow  \forall x, RD_t(x)$
  \item On définit une infinité d'axiomes représentants l'impossibilité d'être sur deux rives en même temps:\\
  $a_5 := \forall x ,RG_t(x) \leftrightarrow \neg RD_t(x)$
  \item Le Loup et le Fermier n'ont pas de prédateurs ils sont donc tout le temps en sécurité\\
        $a_6 := S_t(f) \wedge S_t(l)$
  \item On définit les changements induit par un voyage de la rive gauche vers la rive droite\\
  Soit $x$ l'entité qui change de bord dans cet itération (si le fermier traverse seul alors $x =f$)\\
  Soit $n \in \N$\\
  pour $a_7:= t=2n+1,\\ (\forall y, (\neg (y = x \vee y = f) \wedge RD_{t-1}(y)) \to RD_{t}(y)) \wedge (\forall z,(RG_{t-1}(z) \wedge \neg(x=z \vee z=f)) \rightarrow RG_t(z)) \wedge RD_t(f) \wedge RD_t(x)$\\
  \item On définit les changements induit par un voyage de la rive droite vers la rive gauche\\
  Soit $x$ l'entité qui change de bord dans cet itération (si le fermier traverse seul alors $x =f$)\\
  Soit $n \in \N^*$\\
  pour $a_8 := t=2n,\\ (\forall y, (\neg (y = x \vee y = f) \wedge RG_{t-1}(y)) \to RG_{t}(y)) \wedge (\forall z,(RD_{t-1}(z) \wedge \neg(x=z \vee z=f)) \rightarrow RD_t(z)) \wedge RG_t(f) \wedge RG_t(x)$\\

\end{itemize}

\section{Problème et étude}
\subsection{Problème}


% \section*{Graphe des chemins possibles }

% %:-+-+-+- Engendré par : http://math.et.info.free.fr/TikZ/Arbre/
% \begin{center}
%   % Racine en Haut, développement vers le bas
%   \begin{tikzpicture}[xscale=0.9,yscale=0.9]
%     % Styles (MODIFIABLES)
%     \tikzstyle{fleche}=[->,>=latex,thick]
%     \tikzstyle{noeud}=[fill=white,circle,draw]
%     \tikzstyle{feuille}=[fill=red!20,circle,draw]
%     \tikzstyle{feuilleF}=[fill=lime!20,circle,draw]
%     \tikzstyle{feuilleR}=[fill=orange!20,circle,draw]
%     \tikzstyle{etiquette}=[midway,fill=white,draw]
%     % Dimensions (MODIFIABLES)
%     \def\DistanceInterNiveaux{2.5}
%     \def\DistanceInterFeuilles{2}
%     % Dimensions calculées (NON MODIFIABLES)
%     \def\NiveauA{(-0)*\DistanceInterNiveaux}
%     \def\NiveauB{(-1.8571428571428572)*\DistanceInterNiveaux}
%     \def\NiveauC{(-3.571428571428571)*\DistanceInterNiveaux}
%     \def\NiveauD{(-5.142857142857142)*\DistanceInterNiveaux}
%     \def\NiveauE{(-6.571428571428571)*\DistanceInterNiveaux}
%     \def\NiveauF{(-7.857142857142857)*\DistanceInterNiveaux}
%     \def\NiveauG{(-9)*\DistanceInterNiveaux}
%     \def\NiveauH{(-10)*\DistanceInterNiveaux}
%     \def\InterFeuilles{(1)*\DistanceInterFeuilles}
%     % Noeuds (MODIFIABLES : Styles et Coefficients d'InterFeuilles)
%     \node[feuilleR] (R) at ({(3.5)*\InterFeuilles},{\NiveauA}) {$FLCB|$};
%     \node[feuille] (Ra) at ({(0)*\InterFeuilles},{\NiveauB}) {$LCB|F$};
%     \node[feuille] (Rb) at ({(1)*\InterFeuilles},{\NiveauB}) {$CL|FB$};
%     \node[feuille] (Rc) at ({(2)*\InterFeuilles},{\NiveauB}) {$CB|FL$};
%     \node[noeud] (Rd) at ({(5)*\InterFeuilles},{\NiveauB}) {$LB|FC$};
%     \node[noeud] (Rda) at ({(5)*\InterFeuilles},{\NiveauC}) {$FLB|C$};
%     \node[noeud] (Rdaa) at ({(4)*\InterFeuilles},{\NiveauD}) {$B|FCL$};
%     \node[feuille] (Rdaaa) at ({(3)*\InterFeuilles},{\NiveauE}) {$BF|CL$};
%     \node[noeud] (Rdaab) at ({(4.5)*\InterFeuilles},{\NiveauE}) {$BFC|L$};
%     \node[feuille] (Rdaaba) at ({(4)*\InterFeuilles},{\NiveauF}) {$BC|FL$};
%     \node[noeud] (Rdaabb) at ({(5)*\InterFeuilles},{\NiveauF}) {$C|LFB$};
%     \node[noeud] (Rdaabba) at ({(5)*\InterFeuilles},{\NiveauG}) {$FC|LB$};
%     \node[feuilleF] (Rdaabbaa) at ({(5)*\InterFeuilles},{\NiveauH}) {$|FCLB$};
%     \node[noeud] (Rdab) at ({(6.5)*\InterFeuilles},{\NiveauD}) {$L|FCB$};
%     \node[feuille] (Rdaba) at ({(7)*\InterFeuilles},{\NiveauE}) {$LF|CB$};
%     \node[noeud] (Rdabb) at ({(6)*\InterFeuilles},{\NiveauE}) {$FCL|B$};
%     \node[feuille] (Rdabba) at ({(6)*\InterFeuilles},{\NiveauF}) {$LC|FB$};
%     % Arcs (MODIFIABLES : Styles)
%     \draw[fleche] (R)--(Ra) ;
%     \draw[fleche] (R)--(Rb) ;
%     \draw[fleche] (R)--(Rc) ;
%     \draw[fleche] (R)--(Rd) ;
%     \draw[fleche] (Rd)--(R) ;

%     \draw[fleche] (Rd)--(Rda) ;
%     \draw[fleche] (Rda)--(Rd) ;

%     \draw[fleche] (Rda)--(Rdaa) ;
%     \draw[fleche] (Rdaa)--(Rda) ;
%     \draw[fleche] (Rdab)--(Rda) ;
%     \draw[fleche] (Rdaa)--(Rdaaa) ;
%     \draw[fleche] (Rdaab)--(Rdaa) ;
%     \draw[fleche] (Rdabb)--(Rdab) ;
%     \draw[fleche] (Rdaa)--(Rdaab) ;
%     \draw[fleche] (Rdaabb)--(Rdaab) ;
%     \draw[fleche] (Rdaabb)--(Rdabb) ;
%     \draw[fleche] (Rdaabba)--(Rdaabb) ;

%     \draw[fleche] (Rdaab)--(Rdaaba) ;
%     \draw[fleche] (Rdaab)--(Rdaabb) ;
%     \draw[fleche] (Rdaabb)--(Rdaabba) ;
%     \draw[fleche] (Rdaabba)--(Rdaabbaa) ;
%     \draw[fleche] (Rda)--(Rdab)  ;
%     \draw[fleche] (Rdab)--(Rdaba)  ;
%     \draw[fleche] (Rdab)--(Rdabb) ;
%     \draw[fleche] (Rdabb)--(Rdabba) ;
%     \draw[fleche] (Rdabb)--(Rdaabb) ;
%   \end{tikzpicture}
% \end{center}
% %:-+-+-+-+- Fin


\section*{Existence et Démonstration d'un chemin gagnant}

Dans cette étude une fois que $P_t$ on arrête l'étude de ces chemins.\\
Pour pouvoir appliquer l'axiome 3 on considérera sous entendu que $\forall t, \neg P_{t-1}$ se transmet d'une démonstration à la suivante

\paragraph{\underline{Pour t=0:}} ~~\\
Par Interprétation des états initiaux:\\
Montrons que cet état n'est pas perdant.
\begin{align*}
  RG_0(c) \wedge RG_0(f) \wedge RG_0(l) \wedge RG_0(b) &
  \leftrightarrow RG_0(f) \wedge (RG_0(f) \wedge RG_0(c) \wedge RG_0(b) \wedge RG_0(l))\\
& \leftrightarrow [RG_0(f) \wedge RG_0(f)] \wedge [RG_0(f) \wedge RG_0(c)] \wedge [RG_0(f) \wedge RG_0(b)] \wedge [RG_0(f) \wedge RG_0(l)] \\
& \leftrightarrow \forall x, RG_0(f) \wedge RG_0(x)\\
& \leftrightarrow \forall x, S_0(x) \text{ par axiome 1} \\
& \leftrightarrow \neg P_0 \text{ par axiome 3}
\end{align*}
Montrons que cet état n'est pas gagnant.
\begin{align*}
  RG_0(c) \wedge RG_0(f) \wedge RG_0(l) \wedge RG_0(b) & \rightarrow \exists x , RG_0(x)                              \\
                                                       & \leftrightarrow \exists x, \neg RD_0(x) \text{ par axiome 5} \\
                                                       & \leftrightarrow \neg (\forall x, RD_0(x))                    \\
                                                       & \leftrightarrow \neg G_0 \text{ par axiome 4}                \\
\end{align*}
Donc le jeu peut continuer.

\paragraph{\underline{Pour t=1:}}~~\\

$t$ étant égal à 1 alors $t$ est impair.\\
Par application de l'axiome 7 on a:
\begin{align*}
  (\forall y, RD_{0}(y) \to RD_{1}(y)) \wedge (\forall z,(RG_{0}(z) \wedge \neg(x=z \vee z=f)) \rightarrow RG_1(z)) \wedge RD_1(f) \wedge RD_1(x)
\end{align*}

Il y a donc 4 cas possibles:\\
Déterminons les conditions qui mène à $P_1$ (1)
\begin{align*}
  \exists y,z, (M(y,z)\wedge (RG_1(y)\wedge RG_1(z))) & \to \neg(\forall w, \neg M(w,z)) \vee \neg (\exists w , M(w,z) \wedge (RG_t(z)\wedge RD_t(w)) \vee \neg (RD_t(z)\wedge RG_t(w))\\
  \leftrightarrow \exists y,z, (M(y,z)\wedge (RG_1(y)\wedge RG_1(z))) &\to \neg S_1(z) \text{ par axiome 1}\\
  \leftrightarrow \exists y,z, (M(y,z)\wedge (RG_1(y)\wedge RG_1(z))) &\to P_1 \text{ par axiome 3}
\end{align*}

Montrons le maintenant pour les cas de notre étude\\
Pour $x=l$
\begin{align*}
  (\forall y, RD_{0}(y) \to RD_{1}(y)) \wedge (\forall z,(RG_{0}(z)) \wedge \neg(x=z \vee z=f)) &\rightarrow RG_1(c)) \wedge RD_1(f) \wedge RD_1(l) \wedge RG_1(b)
\end{align*}

Par interprétation de $M(x,y): M(c,b)$.\\
Et $RG_1(c) \wedge RG_1(b)$\\
Donc cette configuration vérifie (1)\\
Donc par équivalence $P_1$\\
~~\\

De manière analogue on démontre que $(x=b \vee x=f) \to P_1$.\\
On en déduit que le seul état qui n'est pas une défaite est l'état tel que :

$RG_1(c)\wedge RG_1(b) \wedge RD_1(f) \wedge RD_1(l)$\\
~~\\
Montrons que cet état n'est pas gagnant.
\begin{align*}
  RG_1(c)\wedge RG_1(b) \wedge RD_1(f) \wedge RD_1(l) & \rightarrow \exists x , RG_1(x)                              \\
                                                       & \leftrightarrow \exists x, \neg RD_1(x) \text{ par axiome 5} \\
                                                       & \leftrightarrow \neg (\forall x, RD_1(x))                    \\
                                                       & \leftrightarrow \neg G_1 \text{ par axiome 4}                \\
\end{align*}

\newpage
\paragraph{\underline{Pour t=2:}}~~\\

t étant pair
\begin{align*}
    (\forall y, RG_{1}(y) \to RG_{2}(y)) \wedge (\forall z,(RD_{1}(z) \wedge \neg(x=z \vee z=f)) \rightarrow RD_{2}(z)) \wedge RG_2(f) \wedge RG_2(x)
\end{align*}
D'après la démonstration précédente

\begin{align*}
  [RG_1(l) \wedge RG_1(b) \wedge RD_1(f) \wedge RD_1(c)] &\to RG_2(l) \wedge RG_2(b) \wedge RG_2(f) \wedge (RG_2(c) \vee RD_2(c)) \text{ par axiome 8}\\
  \leftrightarrow [RG_1(l) \wedge RG_1(b) \wedge RD_1(f) \wedge RD_1(c)] &\to [(RG_2(l) \wedge RG_2(b) \wedge RG_2(f) \wedge RG_2(c))\\
  &\hphantom{aaaaa}\vee (RG_2(l) \wedge RG_2(b) \wedge RG_2(f) \wedge RD_2(c))]
\end{align*}

Il y a donc 2 cas possibles(clauses du OU):\\
\begin{itemize}
    \item \small Le cas où le fermier décide de revenir en emportant avec lui la chèvre sur la rive gauche: $x$ = c, or cet état est similaire à celui pour t=0 dont on a déjà fait la démonstration\\

    \item Le cas où le fermier repart seul $x=f$:\\
    on a alors :
    \begin{align*}
      RG_2(b) \wedge RG_2(f) \wedge RD_2(c) \wedge RG_2(l) &\leftrightarrow S_2(f) \wedge S_2(l) \wedge (RG_2(b) \wedge RG_2(f) \wedge RD_2(c) \wedge RG_2(l)) \text{par axiome $ 6$}\\
    \end{align*}
Montrons alors que cet état n'est pas une défaite\\
Or par axiome 1 $((RG_2(b) \wedge RG_2(f) \to S_2(b)) \wedge ((M(l,c) \wedge RG_2(l)\wedge RD_2(c)) \to S_2(c))$

\begin{align*}
  &\to  S_2(f) \wedge S_2(l) \wedge S_2(b) \wedge S_2(c)\\
  &\leftrightarrow \forall y, S_2(y)\\
  &\leftrightarrow \neg P_2 \text{par axiome $ 3$}
\end{align*}

\end{itemize}
    Non victoire : $RG_2(b) \to \neg G_2$ par axiome 2


\paragraph{\underline{Pour $t$=3}}~~\\
$t$ étant égal à 3 alors $t$ est impair.\\
Par application de l'axiome $ 7$ on a
\begin{align*}
  (\forall y, RD_{2}(y) \to RD_{3}(y)) \wedge (\forall z,(RG_{2}(z)) \wedge \neg(x=z \vee z=f)) \rightarrow RG_3(z)) \wedge RD_3(f) \wedge RD_3(x)
\end{align*}

D'après la démonstration précédente (pour tout état similaire à $t$=2)\\
Pour faciliter le lecture nous noterons X  pour remplacer $RG_2(f) \wedge RG_2(l) \wedge RG_2(b) \wedge RD_2(c)$
\begin{align*}
    RG_2(f) \wedge RG_2(l) \wedge RG_2(b) \wedge RD_2(c)& \to RD_3(c) \wedge RD_3(f) \wedge ((RD_3(l)\wedge RG_3(b))\vee (RD_3(b)\wedge RG_3(l))) \text{par axiome 7}\\
    \leftrightarrow  X& \to \hphantom{aa} (RD_3(c)\wedge RD_3(f)\wedge (RG_3(l)\wedge RD_3(b)))\\
    & \hphantom{aaa}\vee (RD_3(c)\wedge RD_3(f)\wedge (RD_3(b)\wedge RG_3(l)))\\
    & \hphantom{aaa}\vee (RD_3(c) \wedge RG_3(f) \wedge (RG_3(b) \wedge RG_3(l)))\\
    \leftrightarrow  X& \to  S_3(f) \wedge S_3(l) \wedge [(RD_3(c)\wedge RD_3(f)\wedge (RD_3(b)\wedge RG_3(l)))  \\
    & \hphantom{aaaaaaaaaaaaaaa} \vee (RD_3(c)\wedge RD_3(f)\wedge (RD_3(b)\wedge RG_3(l)))\\
    & \hphantom{aaaaaaaaaaaaaaa}\vee (RD_3(c) \wedge RD_3(f) \wedge RG_3(b) \wedge RG_3(l))]\text{ par axiome } 6\\
     \leftrightarrow  X& \to S_3(f) \wedge S_3(l) \wedge ((S_3(c)\wedge S_3(b))\\
    &\hphantom{aaaaaaaaaaaaaaaa}\vee (RD_3(c)\wedge RD_3(f)\wedge (RD_3(b)\wedge RG_3(l)))\\
    &\hphantom{aaaaaaaaaaaaaaaa}\vee (RD_3(c) \wedge RD_3(f) \wedge RG_3(b) \wedge RG_3(l)))\text{ par axiome }  1\\
    \leftrightarrow  X& \to S_3(f) \wedge S_3(l) \wedge [(S_3(c)\wedge S_3(b)) \vee (S_3(b)\wedge S_3(c)) \vee (S_3(c)\wedge S_3(b))]\text{ par axiome }  1\\
    \leftrightarrow  X& \to S_3(f) \wedge S_3(l) \wedge (S_3(c)\wedge S_3(b))\\
    \leftrightarrow  X& \to \forall x, S_3(x)\\
    \leftrightarrow  X& \to \neg P_3 \text{ par axiome }  3
\end{align*}
\newpage
Vérifions si ces chemins sont des victoires:
\begin{itemize}
  \item pour le premier chemin, $RG_3(l) \to \neg G_3$ par axiome 4
  \item pour le second chemin,$RG_3(b) \to \neg G_3$ par axiome 4
  \item pour le dernier chemin(retour en arrière),$RG_3(l) \to \neg G_3$ par axiome 4
\end{itemize}
Aucun chemin ne mène à une défaite, un chemin(une clause du OU) ne sera pas traité car il retourne une situation similaire à celle de t=2\\
~~\\

\paragraph{\underline{Pour $t$=4:}}~~\\
D'après la démonstration précédente (pour t=3 ou autre état similaire):
$RG_3(l) \wedge RG_3(b) \wedge RD_3(f) \wedge RD_3(c)$\\
~~\\
De l'état t=3 (ou similaire) on peut trouver deux nouveaux états qui ne sont ni perdants ni gagnants et qui ne sont pas des retours en arrière,ils forment deux branches distinctes\\
~~\\
t étant pair
\begin{align*}
    (\forall y, RG_{3}(y) \to RG_{4}(y)) \wedge (\forall z,(RD_{3}(z) \wedge \neg(x=z \vee z=f)) \rightarrow RD_{4}(z)) \wedge RG_4(f) \wedge RG_4(x) \text{ par axiome 8}
\end{align*}

\begin{enumerate}
  \item Première branche possible:\\
  Identifions les chemins possibles.\\
  Pour faciliter le lecture nous noterons X  pour remplacer: $ RG_3(b) \wedge RD_3(f) \wedge RD_3(c) \wedge RD_3(l)$
\begin{align*}
       X&\to RG_4(b) \wedge RG_4(f) \wedge ((RD_4(l)\wedge RD_4(c))\vee (RD_4(l)\wedge RG_4(c))\vee(RG_4(l)\wedge RD_4(c)))\\
      \leftrightarrow  X &\to \hphantom{|a} (( RG_4(f) \wedge RG_4(b) \wedge RD_4(l) \wedge RD_4(c))\text{par distributivité du $\wedge$ sur le $\vee$}\\
      & \hphantom{aaa} \vee (RG_4(f) \wedge RG_4(b) \wedge RD_4(l) \wedge RG_4(c))\\
      & \hphantom{aaa} \vee (RG_4(f) \wedge RG_4(b) \wedge RG_4(l) \wedge RD_4(c)))\text{forme normale disjonctive}\\
      \leftrightarrow  X &\to S_4(f) \wedge S_4(l) \wedge (( RG_4(f) \wedge RG_4(b) \wedge RD_4(l) \wedge RD_4(c))\\
      & \hphantom{aaaaaaaaaa|aaaa} \vee (RG_4(f) \wedge RG_4(b) \wedge RD_4(l) \wedge RG_4(c))\\
      & \hphantom{aaaaaaaaaa|aaaa} \vee (RG_4(f) \wedge RG_4(b) \wedge RG_4(l) \wedge RD_4(c)))\text{par axiome 6}\\
\end{align*}
Chaque disjonction représentant un chemin possible démontrons les quelles mènent à une défaite
  \begin{enumerate}
    \item Dans l'état : $RG_4(f) \wedge RG_4(b) \wedge RD_4(l) \wedge RD_4(c)$
    \begin{align*}
      &[M(l,c) \to (\neg(\forall y, \neg M(y,c))\wedge (M(l,c)\wedge RD_4(c)\wedge \neg RG_4(l)) \\
      &\hphantom{aa}  \wedge ((RG_4(f) \wedge RD_4(c))\to \neg ((RG_4(c) \wedge RG_4(f))\vee (RD_4(c)\wedge RD_4(f))))]\\
      &\leftrightarrow \neg S_4(c) \text{par axiome 1}\\
      &\leftrightarrow \exists y,\neg S_4(y)\\
      &\leftrightarrow P_4
    \end{align*}

    Montrons que cet état n'est pas une victoire : $P_4 \to \neg G_4$ par axiome 2
    \item Dans l'état : $RG_4(f) \wedge RG_4(b) \wedge RD_4(l) \wedge RG_4(c)$
    \begin{align*}
      &(RG_4(c)\wedge RG_4(f)\wedge RG_4(b) \to (S_4(b) \wedge S_4(c))) \text{ par axiome 1 (une clause de sécurité est vérifiée)}\\
      &\leftrightarrow (RG_4(c)\wedge RG_4(f)\wedge RG_4(b) \to (S_4(b) \wedge S_4(c))) \wedge S_4(f) \wedge S_4(l)\text{par axiome 6}\\
      &\leftrightarrow \forall y,S_4(y)\\
      &\leftrightarrow \neg P_4
    \end{align*}

    Montrons que cet état n'est pas une victoire: $RG_4(c) \to \neg G_4$ par axiome 2
  \end{enumerate}
  \newpage
  \item Seconde branche possible:\\
  Identifions les chemins possibles.\\
  Pour faciliter le lecture nous noterons X  pour remplacer: $RG_3(l) \wedge RD_3(f) \wedge RD_3(b) \wedge RD_3(c)$
  \begin{align*}
         X&\to RG_4(l) \wedge RG_4(f) \wedge ((RD_4(b)\wedge RD_4(c))\vee (RD_4(b)\wedge RG_4(c))\vee(RG_4(b)\wedge RD_4(c)))\\
        \leftrightarrow  X &\to \hphantom{|a} (( RG_4(f) \wedge RG_4(l) \wedge RD_4(b) \wedge RD_4(c))\text{ par distributivité du $\wedge$ sur le $\vee$}\\
        & \hphantom{aaa} \vee (RG_4(f) \wedge RG_4(l) \wedge RD_4(b) \wedge RG_4(c))\\
        & \hphantom{aaa} \vee (RG_4(f) \wedge RG_4(l) \wedge RG_4(b) \wedge RD_4(c)))\text{ forme normale disjonctive}
  \end{align*}
  \begin{enumerate}
    \item Dans l'état : $RG_4(l) \wedge RG_4(f) \wedge RD_4(b) \wedge RD_4(c)$
    \begin{align*}
      &[M(c,b) \to (\neg(\forall y, \neg M(y,b))\wedge (M(c,b)\wedge RD_4(c)\wedge \neg RG_4(b)) \\
      &\hphantom{aa}  \wedge ((RG_4(f) \wedge RD_4(b))\to \neg ((RG_4(b) \wedge RG_4(f))\vee (RD_4(b)\wedge RD_4(f))))]\\
      &\leftrightarrow \neg S_4(b) \text{par axiome 1}\\
      &\leftrightarrow \exists y,\neg S_4(y)\\
      &\leftrightarrow P_4
    \end{align*}

    Montrons que cet état n'est pas une victoire : $P_4 \to \neg G_4$ par axiome 2
    \item Dans l'état : $RG_4(f) \wedge RG_4(b) \wedge RD_4(l) \wedge RG_4(c)$
    \begin{align*}
      &((RG_4(c)\wedge RG_4(f)) \to  S_4(c)) \wedge ((M(c,b) \wedge RG_4(c)\wedge RD_4(b))\to S_4(b)) \text{ par axiome 1 (une clause de sécurité est vérifiée)}\\
      &\leftrightarrow ((RG_4(c)\wedge RG_4(f)) \to  S_4(c)) \wedge ((M(c,b) \wedge RG_4(c)\wedge RD_4(b))\to S_4(b)) \wedge S_4(f) \wedge S_4(l)\text{par axiome 6}\\
      &\leftrightarrow \forall y,S_4(y)\\
      &\leftrightarrow \neg P_4
    \end{align*}

    Montrons que cet état n'est pas une victoire : $RG_4(c) \to \neg G_4$ par axiome 2
  \end{enumerate}
\end{enumerate}

\paragraph{\underline{Pour $t$=5:}}~~\\
D'après la démonstration précédente (pour t=4 ou autre état similaire)\\
t étant impair
\begin{align*}
    (\forall y, RD_{4}(y) \to RD_{5}(y)) \wedge (\forall z,(RG_{4}(z) \wedge \neg(x=z \vee z=f)) \rightarrow RG_5(z)) \wedge RD_5(f) \wedge RD_5(x)
\end{align*}
\begin{enumerate}
\item Première branche possible
  \begin{align*}
        RG_4(b) \wedge RG_4(f) \wedge RG_4(c) \wedge RD_4(l)&\to RD_5(l) \wedge RD_5(f) \wedge ((RG_5(b)\wedge RD_5(c))\\
        & \hphantom{aaaaa} \vee (RD_5(b)\wedge RG_5(c)) \vee(RG_5(b)\wedge RD_5(c)))\\
        \leftrightarrow RG_4(b) \wedge RG_4(f) \wedge RG_4(c) \wedge RD_4(l) &\to \hphantom{a|} (( RD_5(l) \wedge RD_5(f) \wedge RG_5(b) \wedge RG_5(c))\text{distributivité du $\wedge$ sur le $\vee$}\\
        & \hphantom{aaa} \vee (RD_5(l) \wedge RD_5(f) \wedge RD_5(b) \wedge RG_5(c))\\
        & \hphantom{aaa} \vee (RD_5(l) \wedge RD_5(f) \wedge RD_5(c) \wedge RG_5(b)))\text{forme normale disjonctive}\\
  \end{align*}


Parmis ces trois chemins on ne prendra pas en compte le chemin où le fermier ne part qu'avec la chèvre car c'est un retour à un état similaire à t = 3

Il n'est pas nécessaire de redémontrer les cas possibles ici, en effet par symétrie de $I(S_t)$, échanger les berges d'un état ne change pas les valeur de sécurité des entités dans ce dernier.\\
Dans nos cas:
\begin{enumerate}
  \item  l'état $RG_5(b) \wedge RG_5(c) \wedge RD_5(f) \wedge RD_5(l)$ est le symétrique du cas a) de la branche deux de l'itération\\t = 4 qui donnait $\neg S_4(b)$\\
   D'où
  \begin{align*}
    RG_5(b) \wedge RG_5(c) \wedge RD_5(f) \wedge RD_5(l) &\to \neg S_5(b)\\
    \leftrightarrow RG_5(b) \wedge RG_5(c) \wedge RD_5(f) \wedge RD_5(l) &\to \exists y, \neg S_5(y)\\
    \leftrightarrow RG_5(b) \wedge RG_5(c) \wedge RD_5(f) \wedge RD_5(l) &\to P_5 \text{ par axiome 3}
  \end{align*}
  Et $P_5 \to \neg G_5$ par axiome 2
  \item  l'état $RG_5(c) \wedge RD_5(b) \wedge RD_5(f) \wedge RD_5(l)$ est le symétrique de l'état résultant de t=2 qui donnait $\forall y, S_2(y)$
   D'où
  \begin{align*}
    RG_5(c) \wedge RD_5(b) \wedge RD_5(f) \wedge RD_5(l) &\to \forall y, S_2(y)\\
    \leftrightarrow RG_5(c) \wedge RD_5(b) \wedge RD_5(f) \wedge RD_5(l) &\to \neg P_5 \text{ par axiome 3}
  \end{align*}
  Et $RG_5(c) \to \neg G_5$ par axiome 2
\end{enumerate}

\item Seconde branche possible
  \begin{align*}
        RG_4(l) \wedge RG_4(f) \wedge RG_4(c) \wedge RD_4(b)&\to RD_5(b) \wedge RD_5(f) \wedge ((RG_5(l)\wedge RD_5(c))\\
        & \hphantom{aaaaa} \vee (RD_5(l)\wedge RG_5(c)) \vee(RG_5(l)\wedge RG_5(c)))\\
        \leftrightarrow RG_4(l) \wedge RG_4(f) \wedge RG_4(c) \wedge RD_4(b) &\to (( RD_5(b) \wedge RD_5(f) \wedge RG_5(l) \wedge RG_5(c))\text{par distributivité du $\wedge$ sur le $\vee$}\\
        & \hphantom{aaaaa} \vee (RD_5(b) \wedge RD_5(f) \wedge RD_5(l) \wedge RG_5(c))\\
        & \hphantom{aaaaa} \vee (RG_5(l) \wedge RD_5(f) \wedge RD_5(b) \wedge RD_5(c)))\text{forme normale disjonctive}\\
  \end{align*}


Parmis ces trois chemins on ne prendra pas en compte le chemin où le fermier ne part qu'avec la chèvre car c'est un retour à un état similaire à t = 3

Il n'est pas nécessaire de redémontrer les cas possible ici, en effet par symétrie de la définition de $I(S_t)$, échanger les berges d'un état ne change pas les valeur de sécurité des entités dans ce dernier.\\Dans nos cas:
\begin{enumerate}
  \item  l'état $RG_5(l) \wedge RG_5(c) \wedge RD_5(f) \wedge RD_5(b)$ est le symétrique du cas a) de la branche 1 de l'itération t = 4 qui donnait $\neg S_4(c)$\\
   D'où
  \begin{align*}
    RG_5(l) \wedge RG_5(c) \wedge RD_5(f) \wedge RD_5(b) &\to \neg S_5(c)\\
    \leftrightarrow RG_5(l) \wedge RG_5(c) \wedge RD_5(f) \wedge RD_5(b) &\to \exists y, \neg S_5(y)\\
    \leftrightarrow RG_5(l) \wedge RG_5(c) \wedge RD_5(f) \wedge RD_5(b) &\to P_5 \text{ par axiome 3}
  \end{align*}
  Et $P_5 \to \neg G_5$ par axiome 2
  \item  l'état $RG_5(c) \wedge RD_5(b) \wedge RD_5(f) \wedge RD_5(l)$ est le symétrique de l'état résultant de t=2 qui donnait $\forall y, S_2(y)$
   D'où
  \begin{align*}
    RG_5(c) \wedge RD_5(b) \wedge RD_5(f) \wedge RD_5(l) &\to \forall y, S_2(y)\\
    \leftrightarrow RG_5(c) \wedge RD_5(b) \wedge RD_5(f) \wedge RD_5(l) &\to \neg P_5 \text{ par axiome 3}
  \end{align*}
  Et $RG_5(c) \to \neg G_5$ par axiome 2
\end{enumerate}
\end{enumerate}

\paragraph{\underline{Pour $t$=6:}}~~\\
t étant pair
\begin{align*}
    (\forall y, RG_{5}(y) \to RG_{6}(y)) \wedge (\forall z,(RD_{5}(z) \wedge \neg(x=z \vee z=f)) \rightarrow RD_{6}(z)) \wedge RG_6(f) \wedge RG_6(x)
\end{align*} par axiome $ 8$\\

D'après la démonstration précédente (pour t=5 ou autre état similaire), 3 états sont disponnible depuis t=5.\\ Par formule précédente:
\begin{align*}
  RG_5(c) \wedge RD_5(l) \wedge RD_5(f) \wedge RD_5(b)&\to RG_6(c) \wedge RG_6(f) \wedge ((RD_6(l)\wedge RD_6(b))\\
  & \hphantom{aaaaa} \vee (RD_6(l)\wedge RG_6(b)) \vee(RG_6(l)\wedge RD_6(b)))\\
  \leftrightarrow RG_5(c) \wedge RD_5(l) \wedge RD_5(f) \wedge RD_5(b) &\to\hphantom{a|} (( RD_6(b) \wedge RD_6(l) \wedge RG_6(f) \wedge RG_6(c))\text{ par distributivité du $\wedge$ sur le $\vee$}\\
  & \hphantom{aaa} \vee (RD_6(l) \wedge RG_6(f) \wedge RG_6(b) \wedge RG_6(c))\\
  & \hphantom{aaa} \vee (RG_6(l) \wedge RG_6(f) \wedge RD_6(b) \wedge RG_6(c))\text{forme normale disjonctive}\\
\end{align*}

Les cas $RD_6(l) \wedge RG_6(f) \wedge RG_6(b) \wedge RG_6(c)$ et $RG_6(l) \wedge RG_6(f) \wedge RD_6(b) \wedge RG_6(c)$ sont des retours en arrière vers des états similaire à ceux de t=5. Donc ils ne seront pas redémontrer.

Et le cas $RD_6(b) \wedge RD_6(l) \wedge RG_6(f) \wedge RG_6(c)$ est le symétrique de l'unique état en t = 1 qui ne donne pas une défaite. et il résultait pour cet état que  $\forall y, S_1(y)$

 D'où
\begin{align*}
  RD_6(b) \wedge RD_6(l) \wedge RG_6(f) \wedge RG_6(c) &\to \forall y, S_6(y)\\
  \leftrightarrow RD_6(b) \wedge RD_6(l) \wedge RG_6(f) \wedge RG_6(c) &\to \neg P_6 \text{ par axiome 3}
\end{align*}
Et $RG_6(c) \to \neg G_6$ par axiome 2

\paragraph{\underline{Pour $t$=7:}}

D'après la démonstration précédente (pour t=6 ou autre état similaire)\\
t étant impair
\begin{align*}
    (\forall y, RD_6(y) \to RD_{7}(y)) \wedge (\forall z,(RG_6(z) \wedge \neg(x=z \vee z=f)) \rightarrow RG_7(z)) \wedge RD_7(f) \wedge RD_7(x)
\end{align*}

Déterminons les chemins possibles.
\begin{align*}
  RG_6(c) \wedge RD_6(l) \wedge RG_6(f) \wedge RD_6(b) &\to ((RD_6(b) \wedge RD_6(l) \wedge RD_6(f) \wedge RD_6(c)) \vee (RD_6(b) \wedge RD_6(l) \wedge RD_6(f) \wedge RG_6(c)))\\
\end{align*}

Or le cas $RD_6(b) \wedge RD_6(l) \wedge RD_6(f) \wedge RG_6(c)$ est un retour à un état similaire à la démonstration de t=6, nous ne le redémontrerons pas. \\

Et le cas $RD_6(b) \wedge RD_6(l) \wedge RD_6(f) \wedge RD_6(c)$ est le symétrique aux cas similaires à t = 0 qui donnait $\forall y, S_0(y)$

 D'où
\begin{align*}
  RD_7(b) \wedge RD_7(l) \wedge RD_7(f) \wedge RD_7(c) &\to \forall y, S_7(y)\\
  \leftrightarrow RD_7(b) \wedge RD_7(l) \wedge RD_7(f) \wedge RD_7(c) &\to \neg P_7 \text{ par axiome 3}
\end{align*}

De plus $RD_7(b) \wedge RD_7(l) \wedge RD_7(f) \wedge RD_7(c) \wedge \neg P_7 \to G_7$ par axiomes 2 et 4
\\Cet état est une victoire nous pouvons arrêter l'étude ici.
\section{Conclusion}

La logique choisie permet bien de démontrer l'existence d'une solution à cette énigme et une application étape par étape permet de déterminer quels chemins mennent à un echec et lesquels aboutissent .\\
 En adaptant légèrement cette logique, notamment autour du prédicat M(x,y) et des axiomes 7 et 8, il serait tout aussi possible d'établir une preuve de terminaison d'autre énigmes similaire telles que l'énigme des quatres maris jalloux. Les mêmes synthaxes, sémantiques et définitions sont utilisables mais il faudrait adapter certains prédicats et axiomes.

\end{document}
